\chapter{\abstractname}

%TODO: Abstract
This thesis topic presents design, implementation and evaluation of real-time speaker recognition system on conversation, recorded by a smart phone to detect voice of its owner. The findings and deliverable would become basis for future work in smart mobile interruption management.

The project is implemented as an Android app and works in two modes : in \textit{autonomous mode} processing is performed on smart phone only. In \textit{server mode} recognition is done by transmitting extracted audio features to a server and receiving classification results back. We evaluated both modes with special attention given to investigating battery consumption, phone CPU load, CPU time, memory accumulated, recognition performance and recognition delay. One of the implicit aim is also to understand feasibility of executing audio analysis processes on android in \textit{autonomous mode}. 

The application was tested in different real-life environments and in a full-day evaluation study. The application runs with a fully charged battery up to 13.75 h on a Samsung Galaxy S3 and up to 20.30 h on a Xiaomi Mi3 phone. The speaker recognition system reached an identification accuracy of 75\% for 5 speakers in meeting room conditions. In other daily life situations it reached accuracies from 60\% to 84\%.

